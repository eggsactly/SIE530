\documentclass[12pt]{article}
\usepackage{lingmacros}
\usepackage{tree-dvips}
\usepackage{amsmath}
\usepackage{accents}
\usepackage[hidelinks]{hyperref}

\newcommand{\ubar}[1]{\underaccent{\bar}{#1}}

\setlength\parindent{0pt}

\begin{document}

\title{SIE 530 Comprehensive Notes}
\date{September 23, 2017}
\maketitle

\section*{Definitions}

\begin{itemize}

\item Disjoint: Event $A$ and event $B$ are disjoint, or mutually exclusive, if $A$ and $B$ have no outcome in common $A \cap B = \emptyset \leftrightarrow$ $A$ and $B$ are disjoint events. \cite[p.18]{classnotes.1}

\item Event: A collection of elements contained in the \emph{sample space} \cite[p.16]{classnotes.1}

\item Exhaustive: Event $A$ and event $B$ are exhaustive if the union is $S$. If $A \cap B = \emptyset \leftrightarrow$ $A$ and $B$ are exhaustive events. \cite[p.18]{classnotes.1}

\item Median: A value such that at least 50\% of the data values are at of below this value and at least 50\% of the data values are at or above this value. (Not sensitive to the outlier). \cite[p.9]{classnotes.1}

\item Population: A finite well defined group, which can be enumerated in theory. \cite[p.9]{classnotes.1}

\item Random Experiment: an action or process whos outcome is uncertain \cite[p.16]{classnotes.1}

\item Sample: A subset of a population obtained through a process (possibly random selection). \cite[p.9]{classnotes.1}

\item Sample Space ($S$): A collection of all possible outcomes of a \emph{Random Experiment} \cite[p.16]{classnotes.1}

\end{itemize}

\newpage

\section*{Basic Statistics}

Central Tendency: sample average/mean \cite[p.14]{classnotes.1}.
$$\bar{x}=\frac{\sum\limits_{i=1}^{n} x_i}{n}$$

Scatter/dispersion: \emph{sample variance} or \emph{sample standard deviation} \cite[p.14]{classnotes.1}.

$$\hat{\sigma}=S=\sqrt{\frac{\sum\limits_{i=1}^{n}(x_i-\bar{x})^2}{n-1}}$$

\subsection*{Set Notation}
\begin{itemize}

\item $s\in S$: The outcome $s$ belongs to the sampel space $S$ \cite[p.17]{classnotes.1}

\item $\emptyset=\{\}$: The empty set, the set of no elements. Defines the set of elements of an impossible event \cite[p.17]{classnotes.1}

\item Union: The union of event $A$ and event $B$ denonted $A \cup B$, is the collection (or set) of elements that belong to \emph{either} $A$ or $B$ or both \cite[p.17]{classnotes.1}

\item Intersection: The intersection of event $A$ and event $B$ denoted $A \cap B$ is the collection (or set) of elements that belong to $A$ and $B$ \cite[p.17]{classnotes.1}

\item Complementation: collection of elements that do not belong to $A$. $A^c=\{x:x\notin A\}$ \cite[p.17]{classnotes.1}

\item $A\subset B$: The event course-references.bib$A$ is contained in event $B$, $A\subset B$, if every element of $A$ also belonegs to $B$ \cite[p.17]{classnotes.1}

\end{itemize}

\subsection*{Properties of Sets \cite[p.19]{classnotes.1}}

\begin{itemize}

\item Commutativity: $A \cup B = B \cup A; A \cap B = B \cap A$

\item Associativity: $A\cup (B \cup C)=(A \cup B) \cup C; A \cap (B \cap C) = (A \cap B) \cap C$

\item Distributive Laws: $A \cap(B \cup C) = (A \cap B) \cup ( A \cap C)$ $ A \cup (B \cap C) = (A \cup B) \cap (A \cup C)$

\item DeMorgan's Laws:
$(A \cup B)^c=A^c \cap B^c$
$(A \cap B)^c=A^c\cup B^c$

\end{itemize}

\subsection*{Definition of Probability \cite[p.4]{classnotes.2}}
[P \(\) : \{set of all possible events\} $\rightarrow [0,1]$]
\begin{itemize}

\item Axiom 1: For any event $A, P(A)\geq 0$ (nonnegatove).
\item Axiom 2: $P(S)=1$
\item Axiom 3: Fot any sequence of disjoint sets $A_1,A_2,\dots,A_n, P(A_1 \cup A_2 \cup \dots A_n)=\sum\limits_{i=1}^nP(A_i)$, where n is the total number of disjoint sets in the sequence.

\end {itemize}

\subsection*{Properties and Additive Laws \cite[p.5]{classnotes.2}}
\begin{itemize}
\item $P(A)=1-P(A^c);$
\item $P(\emptyset)=0$
\item $P(S)=1$
\item if $A$ and $B$ are disjoint $P(A\cup B)=0$
\item $A\subset B \rightarrow P(A)\leq (B)$
\item $P(A \sup B) = P(A) + P(B) - P(A \cap B)$
\end{itemize}

Bonferroni's inequality \cite[p.6]{classnotes.2}
$$P(A\cap B) \geq P(A) + P(B) - 1$$

Boole's inequality \cite[p.5]{classnotes.2}: for any sets $A_1, A_2, \dots \subset S$
$$P(\bigcup\limits_{i=1}^\infty\leq \sum\limits_{i=1}^\infty P(A_i)$$

\subsection*{Counting Techniquies}

\begin{itemize}
\item Permutations: Select k objects from a total n without replacement and order matters, what is the number of permutations? \cite[p.9]{classnotes.2}:
$$P_{k,n} = \frac{n!}{(n-k)!}$$
\item Combinations: Select k objects from a total of n without replacement and order does not matter, what is the number of combinations? \cite[p.10]{classnotes.2}
$$C_{k,n}=
\begin{pmatrix}
n \\
k
\end{pmatrix}
=\frac{n!}{k!(n-k)!}
$$
\end{itemize}

\subsection*{Conditional Probability}
Probability of $A$ such that $B$ has occured \cite[p.12]{classnotes.2}:
$$P(A|B)=\frac{P(A \cap B)}{P(B)}$$ 
$$P(A|B)P(B)=P(B|A)P(A)$$
$$P(A_1|B)+P(A_@|B)+\dots P(A_k|B)=1$$
Law of total probability:
$$\sum\limits_{i=1}^k P(B|A_i)P(A_i)=P(B)$$

\subsection*{Bayes Theorem}
Let the events $A_1,A_2,\dots A_k$ be \emph{disjoint} and \emph{exhaustive} events in the sample space $S$, such that $P(A_i)>0$ and let $B$ be an event such that $P(B)>0$, then, \cite[p.7]{classnotes.3}:
$$P(A_i|B)=\frac{P(B|A_i)P(A_i)}{\sum_{j=1}^k P(B|A_i)P(A_j)}$$
$$P(A_i|B)=\frac{P(A_i \cap B}{P(B)} = \frac{cond. \quad prob.}{law \quad of \quad total \quad prob.}$$
\newpage
\bibliographystyle{unsrt}
\bibliography{course-references}

\end{document}
