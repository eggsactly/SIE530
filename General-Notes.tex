\documentclass[12pt]{article}
\usepackage{lingmacros}
\usepackage{tree-dvips}
\usepackage{amsmath}
\usepackage{accents}
\usepackage[hidelinks]{hyperref}

\newcommand{\ubar}[1]{\underaccent{\bar}{#1}}

\setlength\parindent{0pt}

\begin{document}

\title{SIE 530 Comprehensive Notes}
\date{September 23, 2017}
\maketitle

\section*{Definitions}

\begin{itemize}

\item Event: A collection of elements contained in the \emph{sample space} \cite[p.16]{classnotes.1}

\item Median: A value such that at least 50\% of the data values are at of below this value and at least 50\% of the data values are at or above this value. (Not sensitive to the outlier). \cite[p.9]{classnotes.1}

\item Population: A finite well defined group, which can be enumerated in theory. \cite[p.9]{classnotes.1}

\item Random Experiment: an action or process whos outcome is uncertain \cite[p.16]{classnotes.1}

\item Sample: A subset of a population obtained through a process (possibly random selection). \cite[p.9]{classnotes.1}

\item Sample Space ($S$): A collection of all possible outcomes of a \emph{Random Experiment} \cite[p.16]{classnotes.1}

\end{itemize}

\newpage

\section*{Basic Statistics}

Central Tendency: sample average/mean \cite[p.14]{classnotes.1}.
$$\bar{x}=\frac{\sum\limits_{i=1}^{n} x_i}{n}$$

Scatter/dispersion: \emph{sample variance} or \emph{sample standard deviation} \cite[p.14]{classnotes.1}.

$$\hat{\sigma}=S=\sqrt{\frac{\sum\limits_{i=1}^{n}(x_i-\bar{x})^2}{n-1}}$$

\subsection*{Set Notation}
\begin{itemize}

\item $s\in S$: The outcome $s$ belongs to the sampel space $S$ \cite[p.17]{classnotes.1}

\item $\emptyset=\{\}$: The empty set, the set of no elements. Defines the set of elements of an impossible event \cite[p.17]{classnotes.1}

\item Union: The union of event $A$ and event $B$ denonted $A \cup B$, is the collection (or set) of elements that belong to \emph{either} $A$ or $B$ or both \cite[p.17]{classnotes.1}

\item Intersection: The intersection of even $A$ and event $B$ denoted $A \cap B$ is the collection (or setO of elements that beong to $A$ and $B$ \cite[p.17]{classnotes.1}

\end{itemize}

\newpage
\bibliographystyle{unsrt}
\bibliography{course-references}

\end{document}
