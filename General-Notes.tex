\documentclass[12pt]{article}
\usepackage{lingmacros}
\usepackage{tree-dvips}
\usepackage{amsmath}
\usepackage{accents}
\usepackage[hidelinks]{hyperref}

\newcommand{\ubar}[1]{\underaccent{\bar}{#1}}

\setlength\parindent{0pt}

\begin{document}

\title{SIE 530 Comprehensive Notes}
\date{September 23, 2017}
\maketitle

\section*{Definitions}

\begin{itemize}

\item Disjoint: Event $A$ and event $B$ are disjoint, or mutually exclusive, if $A$ and $B$ have no outcome in common $A \cap B = \emptyset \leftrightarrow$ $A$ and $B$ are disjoint events. \cite[p.18]{classnotes.1}

\item Event: A collection of elements contained in the \emph{sample space} \cite[p.16]{classnotes.1}

\item Exhaustive: Event $A$ and event $B$ are exhaustive if the union is $S$. If $A \cap B = \emptyset \leftrightarrow$ $A$ and $B$ are exhaustive events. \cite[p.18]{classnotes.1}

\item Median: A value such that at least 50\% of the data values are at of below this value and at least 50\% of the data values are at or above this value. (Not sensitive to the outlier). \cite[p.9]{classnotes.1}

\item Population: A finite well defined group, which can be enumerated in theory. \cite[p.9]{classnotes.1}

\item Random Experiment: an action or process whos outcome is uncertain \cite[p.16]{classnotes.1}

\item Sample: A subset of a population obtained through a process (possibly random selection). \cite[p.9]{classnotes.1}

\item Sample Space ($S$): A collection of all possible outcomes of a \emph{Random Experiment} \cite[p.16]{classnotes.1}

\end{itemize}

\newpage

\section*{Basic Statistics}

Central Tendency: sample average/mean \cite[p.14]{classnotes.1}.
$$\bar{x}=\frac{\sum\limits_{i=1}^{n} x_i}{n}$$

Scatter/dispersion: \emph{sample variance} or \emph{sample standard deviation} \cite[p.14]{classnotes.1}.

$$\hat{\sigma}=S=\sqrt{\frac{\sum\limits_{i=1}^{n}(x_i-\bar{x})^2}{n-1}}$$

\subsection*{Set Notation}
\begin{itemize}

\item $s\in S$: The outcome $s$ belongs to the sampel space $S$ \cite[p.17]{classnotes.1}

\item $\emptyset=\{\}$: The empty set, the set of no elements. Defines the set of elements of an impossible event \cite[p.17]{classnotes.1}

\item Union: The union of event $A$ and event $B$ denonted $A \cup B$, is the collection (or set) of elements that belong to \emph{either} $A$ or $B$ or both \cite[p.17]{classnotes.1}

\item Intersection: The intersection of event $A$ and event $B$ denoted $A \cap B$ is the collection (or set) of elements that belong to $A$ and $B$ \cite[p.17]{classnotes.1}

\item Complementation: collection of elements that do not belong to $A$. $A^c=\{x:x\notin A\}$ \cite[p.17]{classnotes.1}

\item $A\subset B$: The event $A$ is contained in event $B$, $A\subset B$, if every element of $A$ also belonegs to $B$ \cite[p.17]{classnotes.1}

\end{itemize}

\subsection*{Properties \cite[p.19]{classnotes.1}}

\begin{itemize}

\item Commutativity: $A \cup B = B \cup A; A \cap B = B \cap A$

\item Associativity: $A\cup (B \cup C)=(A \cup B) \cup C; A \cap (B \cap C) = (A \cap B) \cap C$

\item Distributive Laws: $A \cap(B \cup C) = (A \cap B) \cup ( A \cap C)$ $ A \cup (B \cap C) = (A \cup B) \cap (A \cup C)$

\item DeMorgan's Laws:
$(A \cup B)^c=A^c \cap B^c$
$(A \cap B)^c=A^c\cup B^c$

\end{itemize}

\newpage
\bibliographystyle{unsrt}
\bibliography{course-references}

\end{document}
